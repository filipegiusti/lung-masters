\chapter{Segmentação}

Segmentação é a área do processamento de imagens que trata de isolar as regiões de interesse de uma imagem ou mudar a sua representação para facilitar a sua análise em determinada aplicação. Segmentação de imagens é tipicamente usada para localizar objetos e formas (curvas, linhas, etc) em imagens.

A segmentação de imagens não triviais é uma das tarefas mais difíceis no processamento de imagens. A precisão da segmentação determina o sucesso ou falha de um sistema de análise computadorizada \cite{gonzalez}.

Mas o processo de segmentação é enormemente facilitado quando o domínio das imagens do problema é bem conhecido e restrito. Desse forma permitindo que as técnicas de segmentação possam ser alteradas para trazer resultados mais satisfatórios no domínio do problema.

\section{Imagens de tomografia computadorizada}

As imagens de tomografia computadorizada são geradas a partir de diversas imagens de raio-x que são tiradas de fatias finas do paciente em diversos ângulos, como o resultado de cada pixel dessa fatia é a soma das energias bloqueadas pelos tecidos do corpo, e sabe-se exatamente em que ângulo cada fatia foi tirada, pode-se reconstruir a imagem em 2 dimensões dessa fatia. Como são geradas imagens de várias fatias, é possível uma reconstrução em 3 dimensões do corpo do paciente.

Essas imagens são geralmente de 512x512 pixels, embora alguns equipamentos possuam uma resolução de 1024x1024 pixels, cada um deles possuindo 256 níveis de cinza. O tamanho do pixel é da ordem de 1mm, podendo chegar em algumas unidades a valores em torno de 0,1mm.
% TODO: referencia

% qualidade da imagem

\section{Métodos de segmentação}



\subsection{Threshold Adaptativo}



