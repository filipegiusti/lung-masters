\documentclass[tc]{texufpel}

\usepackage[brazilian]{babel}
\usepackage[utf8]{inputenc}
\usepackage[T1]{fontenc}
\usepackage{graphicx}
\usepackage{subfigure}
\usepackage{array}
%\usepackage{abntcite}
\usepackage{times}              % pacote para usar fonte Adobe Times
\usepackage{mathptmx}          % p/ usar fonte Adobe Times nas fórmulas
\usepackage{setspace}
	\onehalfspacing

\author{Giusti}{Filipe Vernetti}
\title{FERRAMENTA PARA AUXÍLIO AO DIAGNÓSTICO DE IMAGENS PULMONARES DE TOMOGRAFIA COMPUTADORIZADA}
\advisor[Prof.~Dr.]{Oliveira}{Lucas Ferrari}

\date{Dezembro}{2008}
%\course{Curso de Especialização em Cachaça}
%\location{Itaquaquecetuba}{SP}}

\renewcommand{\nominata}{
        UNIVERSIDADE FEDERAL DE PELOTAS\\
        Reitor: Prof.~Dr.~Cesar Borges\\
        Coordenador do curso: Prof.~Dr.~Lucas Ferrari Oliveira\\
}

\keyword{processamento de imagens}
\keyword{tomografia computadorizada}
\keyword{pulmão}
% TODO: keywords

\begin{document}

\maketitle

% TODO: examinadores
\examiner{Prof.~Dr.~Armando Multas}
\examiner{Prof\textsuperscript{a}.~Dr\textsuperscript{a}.~Nomelinda Longuinha da Silva Paes Netto}
\examiner{Prof.~MSc.~Gerúndio das Dores}
\makeexaminers

% frase de efeito
% TODO: frase de efeito
\clearpage
\begin{flushright}
\mbox{}\vfill
{\sffamily\itshape
``If I have seen farther than others,\\
it is because I stood on the shoulders of giants.''\\}
--- \textsc{Sir~Isaac Newton}
\end{flushright}

% agradecimentos
% TODO: agradecimentos
\chapter*{Agradecimentos}
Agradeço ao \LaTeX\ por não ter vírus de macro\ldots

% Dedicatoria
% TODO: dedicatoria
\chapter*{Dedicatoria}
Dedico esse trabalho a mim.

% TODO: resumo
\begin{abstract}
Este documento é um exemplo de como formatar documentos para o
Instituto de Física e Matemática da UFPEL usando as classes \LaTeX\@. Ao mesmo tempo, pode servir de consulta
para comandos mais genéricos. \emph{O texto do resumo não deve
conter mais do que 500 palavras.}
\end{abstract}

% TODO: resumo ingles
\begin{englishabstract}{Using \LaTeX\ to Prepare Documents at IFM/UFPEL}{Electronic document preparation, \LaTeX, ABNT, UFPEL}
This document is an example on how to prepare documents at IFM/UFPEL
using the \LaTeX\ classes provided by the UTUG\@. At the same time, it
may serve as a guide for general-purpose commands. \emph{The text in
the abstract should not contain more than 500~words.}
\end{englishabstract}

\listoffigures
\listoftables

\begin{listofabbrv}{SPMD}
        \item[TC] Tomografia Computadorizada
	\item[UH] Unidade Hounsfield
\end{listofabbrv}

%\begin{listofsymbols}{$\alpha\beta\pi\omega$}
%       \item[$\sum{\frac{a}{b}}$] Somatório do produtório
%       \item[$\alpha\beta\pi\omega$] Fator de inconstância do resultado
%\end{listofsymbols}

\tableofcontents

\chapter{Introdução}

O exame de tomografia computadorizada resulta em diversas imagens que representam secções do corpo. Elas são geradas através de uma sucessão de raios-x que são posteriormente processados por computador. Essas imagens podem ser reunidas de forma a se obter uma representação em 3 dimensões do corpo.

Ao analisar um exame de tomografia computadorizada para realizar um laudo, o médico faz um julgamento subjetivo com base na sua experiência. Se o médico possuir pouca experiência e não estiver seguro o suficiente para emitir o laudo, terá de realizar uma busca pelas patologias que são mais prováveis e comparar a descrição encontrada com a imagem do paciente, apesar de que às vezes são encontradas descrições com algumas imagens de exemplo, comparar uma descrição textual com uma imagem é um processo muito suscetível à falhas.

O exame de tomografia computadorizada gera muitas imagens a serem analisadas, e essa grande quantidade de imagens contribui enormemente para o aumento de falhas humanas, pois analisar diversas imagens é um processo trabalhoso, complexo e tedioso. O grande número possível de combinações de padrões complexos achados nas diversas imagens, a falta de correlação fortemente estabelecida entre os achados radiológicos e patológicos e variações na forma de interpretação e descrição dos achados radiológicos, sem uma definição objetiva, são os principais fatores que resultam em grandes variações entre diagnósticos \cite{uchiyama}.

Para tentar acelerar a análise e diminuir o número de laudos incorretos, foi desenvolvida uma ferramenta para auxiliar os médicos na tomada de decisão do laudo. Os resultados alcançados estão expostos nesse trabalho.

A ferramenta é capaz de processar e analisar a imagem do paciente, e então recuperar de uma base de conhecimento, imagens similares, dessa forma possibilitando ao médico a comparação visual com diversas outras imagens de tomografia computadorizada. Das imagens similares é possível acessar o laudo emitido, criando assim mais subsídios para o médico na hora de emitir o diagnóstico.

\section{Motivação}

Tornar a avaliação de imagens de tomografia computadorizada menos subjetiva, facilitando o trabalho de médicos menos experientes. Além de fornecer uma forma rápida de comparar exames similares de diferentes pacientes, tornando mais preciso o diagnóstico. Um dos propósitos de se desenvolver uma ferramenta como essa reside no fato de que, reconhecidamente, a avaliação desse tipo de doença é um dos problemas mais difíceis no diagnóstico médico \cite{doi}, \cite{bick}.

É importante evidenciar que a ferramenta desenvolvida é distribuída na forma de software livre, podendo, por isso, ser redistribuída, usada e modificada.

\section{Objetivos}

O objetivo desse trabalho é a realização de um estudo sobre técnicas de segmentação e extração de características de imagens pulmonares de tomografia computadorizada e métodos para recuperação de imagens baseado em conteúdo, objetivando o desenvolvimento de um software capaz de auxiliar médicos no diagnóstico desse tipo de imagem.

O software desenvolvido realiza 3 etapas principais:
\begin{enumerate}
 \item Segmentação das imagens resultantes da tomografia computadorizada, visando extrair as regiões de interesse, que são os pulmões, e separa-los. Um dos meios que surgiram para auxiliar na segmentação e registro de imagens médicas foi o framework ITK – Insight Segmentation and Registration Toolkit. Com o qual espera-se conseguir um desenvolvimento mais rápido do software \cite{yoo}.
 \item Criação do vetor de características para cada pulmão segmentado na etapa anterior.
 \item Comparação da imagem com outras previamente inseridas no software, utilizando uma função de similaridade.
\end{enumerate}

\section{Organização}

Este trabalho, dividido em sete capítulos, apresenta primeiramente os conceitos estudados durante o seu desenvolvimento, seguido da descrição da ferramenta desenvolvida, os resultados obtidos e as conclusões.

No segundo capítulo, é apresentada uma introdução sobre as imagens de tomografia computadorizada pulmonar, assim como suas características relevantes, e um estudo sobre o método de segmentação utilizada.

No capítulo seguinte, são explicadas técnicas de extração de características.

No quarto capítulo, é demonstrado o funcionamento básico de um sistema de recuperação de imagens baseado em conteúdo.

No quinto capítulo, é descrito o software desenvolvido durante este trabalho, bem como as bibliotecas utilizadas.

No sexto capítulo são exibidos os resultados obtidos pela ferramenta desenvolvida e, no último capítulo, são apresentadas as conclusões e as sugestões de trabalhos futuros.
\chapter{Segmentação}

Segmentação é a área do processamento de imagens que trata de isolar as regiões de interesse de uma imagem ou mudar a sua representação para facilitar a sua análise em determinada aplicação. Segmentação de imagens é tipicamente usada para localizar objetos e formas (curvas, linhas, etc) em imagens.

A segmentação de imagens não triviais é uma das tarefas mais difíceis no processamento de imagens. A precisão da segmentação determina o sucesso ou falha de um sistema de análise computadorizada \cite{gonzalez}.

Mas o processo de segmentação é enormemente facilitado quando o domínio das imagens do problema é bem conhecido e restrito. Desse forma permitindo que as técnicas de segmentação possam ser alteradas para trazer resultados mais satisfatórios no domínio do problema.

\section{Imagens de tomografia computadorizada}

A tomografia era um dos mais importantes métodos de diagnóstico radiológico até a invenção da tomografia computadorizada, na década de 70 do século passado. Sendo um dos primeiros tipos de exame a se beneficiar da popularização da computação.

A utilização de imagens no campo da medicina tem como principal objetivo proporcionar uma avaliação não invasiva dos tecidos e órgãos do corpo humano, tornando possível a verificação de anormalidades causadas por doenças ou
acidentes \cite{oliveira}.

As imagens de tomografia computadorizada são geradas a partir de diversas imagens de raio-x que são tiradas de fatias finas do paciente em diversos ângulos, como o resultado de cada pixel dessa fatia é a soma das energias bloqueadas pelos tecidos do corpo, e sabe-se exatamente em que ângulo cada fatia foi tirada, pode-se reconstruir a imagem em 2 dimensões dessa fatia. Como são geradas imagens de várias fatias, é possível uma reconstrução em 3 dimensões do corpo do paciente.

% coeficiente de Hounsfield

Essas imagens são geralmente de 512x512 pixels, embora alguns equipamentos possuam uma resolução de 1024x1024 pixels, cada um deles possuindo 256 níveis de cinza. O tamanho do pixel é da ordem de 1mm, podendo chegar em algumas unidades a valores em torno de 0,1mm.
% TODO: referencia

\subsection{Padrão DICOM}

A introdução de imagens médicas digitais na década de 70 do século passado e o uso de computadores para processar estas imagens fizeram com que o American College of Radiology (ACR) e a National Electrical Manufacturers Association (NEMA) se juntassem para formar um comitê com o objetivo de criar um método padrão para a transmissão de imagens médicas e das informações associadas a elas.

Este comitê, constituído em 1983, publicou seu primeiro conjunto de padrões, chamado de ACR-NEMA, em 1985, e o segundo em 1988. Até a publicação dos padrões, a maioria dos equipamentos utilizava formatos proprietários para fazer o armazenamento e comunicação de imagens.

Embora as primeiras versões não tenham obtido êxito total na definição de um padrão comum, a terceira versão, publicada em 1993 sob o nome de DICOM (Digital Imaging and Communications in Medicine), conseguiu estabelecer uma forma padronizada de armazenamento e comunicação de imagens médicas e as correspondentes informações associadas.

Com os melhoramentos promulgados por esta terceira versão, o padrão estava pronto tanto para permitir transferência de imagens médicas em um ambiente com múltiplos fabricantes como também para facilitar o desenvolvimento e a expansão dos sistemas de armazenamento e de comunicação e a conexão com os sistemas de informação médica \cite{nema}.

Nos dias de hoje, a maioria dos fabricantes de equipamentos para aquisição de imagens médicas permite que os arquivos sejam exportados nesse formato, além dos formatos proprietários. Assim como a maioria dos softwares de processamento de imagens médicas também apresenta compatibilidade com esse formato.

\section{Threshold Adaptativo}

Threshold é um dos métodos mais simples de segmentação de imagens. A partir de uma imagem em tons de cinza, o threshold pode ser usado para torná-la binária.

Durante o processo de threshold, os pixels são dividos em dois grupos, na forma mais simples de threshold, a divisão é feita dependendo apenas se o valor do pixel é maior ou menor que o valor de threshold. Então um grupo é colorido de preto e o outro de branco, tornando a imagem binária. A divisão dos grupos de pixels pode ser feita com base em mais de um valor de threshold, dessa forma os pixels são divididos entro os que estão dentro de um intervalo formado por 2 valores de threshold e os que não estão.

O parâmetro chave que determina a eficiência de um threshold é o valor de threshold usado. Diversos métodos existem para a escolha do valor de threshold, ele pode ser escolhido manualmente, ou por algum algoritmo que o compute, dependendo da imagem, esse tipo de threshold mais conhecido como thresolhd automático. Um método simples é escolher o valor da media ou da mediana. Geralmente esse algoritmo só irá atingir bons resultados se a imagem de entrada possuir pouco ruído e for uniforme. Uma abordagem um pouco mais sofisticada seria gerar o histograma da intensidade dos pixels da imagem e escolher como threshold o valor de vale.

A técnica de threshold adaptivo consiste em escolher um valor arbitrário de threshold, realizar o threshold e calcular o valor médio dos pixels acima e abaixo do valor de threshold. Então obter um novo threshold, o qual é a média entre esses dois valores. Com o novo threshold o processo se inicia novamente, ele só para quando se atingir a convergência, ou seja, o novo valor for igual ao antigo valor. Este algoritmo garante a convergência até um mínimo local.

% TODO: arrumar referencias
\bibliographystyle{abnt-alf}
\bibliography{referencias}

\end{document}
