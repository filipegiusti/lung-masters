\chapter{Conclusões}

Este trabalho propôs o desenvolvimento de uma ferramenta para auxiliar no diagnóstico de imagens pulmonares de tomografia computadorizada, através da recuperação de imagens similares a imagem em que se deseja diagnosticar.

Atualmente, está ocorrendo uma expansão na aplicação do processamento de imagens à medicina. Os centros de tratamento médico cada vez mais estão utilizando softwares precisos e confiáveis na análise e diagnóstico de diversas doenças. Como a quantidade de dados coletados por um exame de tomografia geralmente é grande, a utilização de técnicas de processamento de imagens para facilitar a visualização das informações tem se tornado essencial.

\section{Dificuldades encontradas}

Para a segmentação a falta de documentação e alta complexidade do ITK, associado a quantidade de detalhes do padrão DICOM não implementado por muitas ferramentas, até mesmo o ITK, fizeram com que o tempo de aperfeiçoamento do algoritmo fosse diminuido.

Além disso a dificuldade em encontrar especialistas dispostos a ajudar na avaliação do trabalho não permitiu a validação do mesmo.

\section{Trabalhos Futuros}

Este trabalho pode seguir sendo aperfeiçoado mesmo após sua conclusão. As seguintes metas podem ser destacadas:
\\* \\*
\begin{itemize}
 \item Validar a ferramenta com um profissional a área médica.
 \item Acesso ao laudo das imagens retornadas da busca.
 \item A portabilidade da ferramenta para outros sistemas operacionais.
 \item Utilizar outro conjuntos de características, por exemplo, a técnica de wavelets.
 \item Modularizar a ferramenta de forma a permitir a integração com aplicativos desenvolvidos para segmentação de outros tipos de imagens.
 \item Indicar nas imagens os possíveis locais de patologias.
 \item Utilizar outras topologias de RNA.
\end{itemize}

