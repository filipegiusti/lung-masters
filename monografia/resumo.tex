\begin{abstract}
Este trabalho apresenta o resultado do desenvolvimento de uma ferramenta que utiliza técnicas de processamento de imagens e redes neurais artificiais com objetivo de facilitar o diagnóstico de exames de tomografia computadorizada pulmonar. A ferramenta faz isso retornando ao usuário imagens de outros exames de tomografia computadorizada que sejam similares ao exame em que deseja-se realizar o diagnóstico. Sistemas assim são chamados de sistemas de recuperação de imagens baseado em conteúdo. Mas para que possa retornar resultados relevantes, a ferramenta precisa da resposta do usuário sobre quais imagens são realmente similares. Para que possa fazer isso ele utiliza-se de redes neurais artificiais com topologia de pró-alimentação e do algoritmo de retro-propagação para o treinamento, dessa forma o sistema é capaz de aprender a similaridade entre duas imagens. A ferramenta possui uma interface web a qual pode ser instalada em um servidor e acessada de qualquer terminal, facilitando assim a utilização.
\end{abstract}


\begin{englishabstract}{Application for computer aided diagnosis of computed tomography lung images}{image processing, computed tomography, artificial neural network, content based image retrieval}

This paper presents the result of the development of a program that uses techniques of image processing and artificial neural networks to facilitate the diagnosis of lung computed tomography examinations. The system retrieves to the user images from others computed tomography scans that are similar to the exam that you want to do the diagnosis. Systems like this are called content based image retrieval. But to return relevant results, the system needs the user's feedback on which images are similar. In order to do this the program is based on artificial neural networks with feedforward topology and back-propagation algorithm for the training, so the system is able to learn the similarity between two images. The system has a web interface which can be installed on a server and accessed from any terminal, thus facilitating the use.
\end{englishabstract}