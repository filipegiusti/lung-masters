\chapter{Resultados}

Utilizando imagens de exames cedidas pelo HCFMRP – USP foi possível realizar a segmentação de diversas imagens e realizar o treinamento da RNA através da ferramenta desenvolvida.

\section{Segmentação}

Quanto a segmentação dos pulmões foram observados poucos casos em que o resultado não pode ser usada na etapa seguinte, e geralmente pelas imagens estarem em condições extremas.

Não foi possível avaliar os resultados da segmentação com a opinião de um especialista.

\section{Recuperação das imagens}

Utilizando as imagens disponíveis, a RNA conseguiu identificar as imagens utilizadas no conjunto de treinamento, mas por causa do pequeno conjunto de imagens de cada doença (várias possuiam só uma ocorrência) não foi possível a realização da etapa de teste para medir o erro da RNA. Assim como não foi possível ter acesso a um especialista para que uma avaliação do resultado final fosse feita.