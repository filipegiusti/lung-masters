\chapter{Introdução}

O exame de tomografia computadorizada resulta em diversas imagens que representam secções do corpo. Elas são geradas através de uma sucessão de raios-x que são posteriormente processados por computador. Essas imagens podem ser reunidas de forma a se obter uma representação em 3 dimensões do corpo.

Ao analisar um exame de tomografia computadorizada para realizar um laudo, o médico faz um julgamento subjetivo com base na sua experiência. Se o médico possuir pouca experiência e não estiver seguro o suficiente para emitir o laudo, terá de realizar uma busca pelas patologias que são mais prováveis e comparar a descrição encontrada com a imagem do paciente, apesar de que às vezes são encontradas descrições com algumas imagens de exemplo, comparar uma descrição textual com uma imagem é um processo muito suscetível à falhas.

O exame de tomografia computadorizada gera muitas imagens a serem analisadas, e essa grande quantidade de imagens contribui enormemente para o aumento de falhas humanas, pois analisar diversas imagens é um processo trabalhoso, complexo e tedioso. O grande número possível de combinações de padrões complexos achados nas diversas imagens, a falta de correlação fortemente estabelecida entre os achados radiológicos e patológicos e variações na forma de interpretação e descrição dos achados radiológicos, sem uma definição objetiva, são os principais fatores que resultam em grandes variações entre diagnósticos \cite{uchiyama}.

Para tentar acelerar a análise e diminuir o número de laudos incorretos, foi desenvolvida uma ferramenta para auxiliar os médicos na tomada de decisão do laudo. Os resultados alcançados estão expostos nesse trabalho.

A ferramenta é capaz de processar e analisar a imagem do paciente, e então recuperar de uma base de conhecimento, imagens similares, dessa forma possibilitando ao médico a comparação visual com diversas outras imagens de tomografia computadorizada. Das imagens similares é possível acessar o laudo emitido, criando assim mais subsídios para o médico na hora de emitir o diagnóstico.

\section{Motivação}

Tornar a avaliação de imagens de tomografia computadorizada menos subjetiva, facilitando o trabalho de médicos menos experientes. Além de fornecer uma forma rápida de comparar exames similares de diferentes pacientes, tornando mais preciso o diagnóstico. Um dos propósitos de se desenvolver uma ferramenta como essa reside no fato de que, reconhecidamente, a avaliação desse tipo de doença é um dos problemas mais difíceis no diagnóstico médico \cite{doi}, \cite{bick}.

É importante evidenciar que a ferramenta desenvolvida é distribuída na forma de software livre, podendo, por isso, ser redistribuída, usada e modificada.

\section{Objetivos}

O objetivo desse trabalho é a realização de um estudo sobre técnicas de segmentação e extração de características de imagens pulmonares de tomografia computadorizada e métodos para recuperação de imagens baseado em conteúdo, objetivando o desenvolvimento de um software capaz de auxiliar médicos no diagnóstico desse tipo de imagem.

O software desenvolvido realiza 3 etapas principais:
\begin{enumerate}
 \item Segmentação das imagens resultantes da tomografia computadorizada, visando extrair as regiões de interesse, que são os pulmões, e separa-los. Um dos meios que surgiram para auxiliar na segmentação e registro de imagens médicas foi o framework ITK – Insight Segmentation and Registration Toolkit. Com o qual espera-se conseguir um desenvolvimento mais rápido do software \cite{yoo}.
 \item Criação do vetor de características para cada pulmão segmentado na etapa anterior.
 \item Comparação da imagem com outras previamente inseridas no software, utilizando uma função de similaridade.
\end{enumerate}

\section{Organização}

Este trabalho, dividido em sete capítulos, apresenta primeiramente os conceitos estudados durante o seu desenvolvimento, seguido da descrição da ferramenta desenvolvida, os resultados obtidos e as conclusões.

No segundo capítulo, é apresentada uma introdução sobre as imagens de tomografia computadorizada pulmonar, assim como suas características relevantes, e um estudo sobre o método de segmentação utilizada.

No capítulo seguinte, são explicadas técnicas de extração de características.

No quarto capítulo, é demonstrado o funcionamento básico de um sistema de recuperação de imagens baseado em conteúdo.

No quinto capítulo, é descrito o software desenvolvido durante este trabalho, bem como as bibliotecas utilizadas.

No sexto capítulo são exibidos os resultados obtidos pela ferramenta desenvolvida e, no último capítulo, são apresentadas as conclusões e as sugestões de trabalhos futuros.